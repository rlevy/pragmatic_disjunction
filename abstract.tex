Semantic acquisition is often cast in terms of children learning to
relate forms to meanings, but it is actually a lifelong process
mediated largely by language, with only indirect connections to the
world.  All languages have devices for giving explicit definitions
(\word{`Wine lover' means `oenophile'}), but such information is often
conveyed indirectly, via presuppositions and other pragmatic
implications. We study one systematic class of such cases:
definitional disjunctions like \word{wine lover or oenophile}.  Such
uses are possible in all the typologically diverse languages we have
studied, suggesting a deep connection with prototypical disjunction, in
which the disjuncts' meanings are generally disjoint rather than
identical. These uses guide our development of a recursive Bayesian
model of linguistic communication in which the agents value sharing
not only information about the world, as in previous models, but also
about their language, and we show that the model's behavior in
different context closely matches attested patterns in real
interactions.

%%% Local Variables: 
%%% mode: latex
%%% TeX-master: "definitional_disjunction"
%%% End: 
