\section{Conclusion}\label{sec:conclusion}

\mynote{This conclusion can be thought of as a place holder; I'm not sure it goes where we want to.}

This paper developed a recursive Bayesian model of communication that
extends previous models and results into social situations in which
the speaker is presumed to be more expert about the relevant domain
than the listener and has an interest in instructing the listener
about the language used in that domain.  We showed that the model is
able to capture an intricate and subtle instance of such language
pedagogy involving definitional disjunctions.  Though space
considerations preclude detailed discussion, we believe that the same
concepts are at work in \tech{definitional copular constructions} like
\word{Wine lovers are oenophiles} and \word{Oenophiles are wine
  lovers}, in which the speaker uses a normally predicational
construction, not to convey contingent information about a term, but
rather to give a definition of it. As with definitional disjunctions,
the basic semantics of the construction seem at odds with such uses,
but we believe they are governed by the same complex social and
linguistic mechanisms that we modeled in this paper.

%%% Local Variables: 
%%% mode: latex
%%% TeX-master: "definitional_disjunction"
%%% End:

