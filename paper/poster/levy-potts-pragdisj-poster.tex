\documentclass[landscape,a0paper,fontscale=0.292]{baposter}

\usepackage{examples-slim}
\usepackage{amsmath}
\usepackage{amssymb}
\usepackage{booktabs}
\usepackage{colortbl}
\usepackage{url}
\usepackage[normalem]{ulem}
\usepackage{schemata}

%=====================================================================
%========================= cross-references ==========================

% Flexible sec/fig/tbl/def cross-refs.
\newcommand{\Secref}[1]{Section~\ref{#1}}
\newcommand{\secref}[1]{section~\ref{#1}}
\newcommand{\dashsecref}[2]{sections~\ref{#1}--\ref{#2}}

\newcommand{\Defref}[1]{Def.~\ref{#1}}
\newcommand{\defref}[1]{def.~\ref{#1}}
\newcommand{\Defrefc}[2]{\Defref{#1}, clause~\ref{#2}}
\newcommand{\defrefc}[2]{\defref{#1}, clause~\ref{#2}}

\newcommand{\Figref}[1]{Figure~\ref{#1}}
\newcommand{\figref}[1]{figure~\ref{#1}}
\newcommand{\dashfigref}[2]{figures~\ref{#1}--\ref{#2}}
\newcommand{\Tabref}[1]{Table~\ref{#1}}
\newcommand{\tabref}[1]{table~\ref{#1}}

% Examples:
\newcommand{\eg}[1]{(\ref{#1})}
\newcommand{\subeg}[2]{(\ref{#1}\ref{#2})}
\newcommand{\dblsubeg}[3]{(\ref{#1}\ref{#2},~\ref{#3})}
\newcommand{\dashsubeg}[3]{(\ref{#1}\ref{#2}--\ref{#3})}

% In-text citations
\newcommand{\posscitet}[1]{\citeauthor{#1}'s~(\citeyear{#1})}
\newcommand{\sposscitet}[1]{\citeauthor{#1}'~(\citeyear{#1})}
\newcommand{\possciteauthor}[1]{\citeauthor{#1}'s}
\newcommand{\spossciteauthor}[1]{\citeauthor{#1}'}
\newcommand{\pgposscitet}[2]{\citeauthor{#1}'s~(\citeyear{#1}:~#2)}
\newcommand{\secposscitet}[2]{\citeauthor{#1}'s~(\citeyear{#1}:~$\S$#2)}
\newcommand{\pgcitealt}[2]{\citealt{#1}:~#2}
\newcommand{\seccitealt}[2]{\citealt{#1}:~$\S$#2}
\newcommand{\pgcitep}[2]{(\citealt{#1}:~#2)}
\newcommand{\seccitep}[2]{(\citealt{#1}:~$\S$#2)}
\newcommand{\pgcitet}[2]{\citeauthor{#1}~(\citeyear{#1}:~#2)}
\newcommand{\seccitet}[2]{\citeauthor{#1}~(\citeyear{#1}:~$\S$#2)}

%=====================================================================
%============================ text styles ============================

\newcommand{\word}[1]{\emph{#1}}
\newcommand{\tech}[1]{\textbf{#1}}
\definecolor{maroon}{HTML}{990000}
\newcommand{\highlight}[1]{{\color{maroon}#1}}

% Gray table cell:
\newcommand{\graycell}[1]{{\cellcolor[gray]{.8}#1}}

%=====================================================================
%============================== judgments ============================

\newcommand{\bad}{\sqz{${}^\ast$}}
\newcommand{\freebad}{${}^\ast$}
\newcommand{\marked}{\sqz{${}^\#$}}
\newcommand{\freemarked}{${}^\#$}

%=====================================================================
%=============================== model ===============================

\newcommand{\set}[1]{\ensuremath{\left\{ #1 \right\}}}
\newcommand{\True}{\texttt{T}}
\newcommand{\False}{\texttt{F}}
\newcommand{\Reals}{\mathbb{R}}
\newcommand{\given}{\mid}
\newcommand{\Indicator}{\mathbb{I}}

\newcommand{\sem}[1]{\ensuremath{\llbracket#1\rrbracket}}
\newcommand{\States}{T}
\newcommand{\state}{t}
\newcommand{\Lex}{\mathcal{L}}
\newcommand{\LexStar}{\Lex^{\ast}}
\newcommand{\LexSet}{\mathbf{L}}
\newcommand{\Messages}{M}
\newcommand{\msg}{m}
\newcommand{\Costs}{C}
\newcommand{\Prior}{P}
\newcommand{\LexPrior}{P_{\LexSet}}

\newcommand{\listenerZero}{l_{0}}
\newcommand{\speakerOne}{s_{1}}
\newcommand{\listenerOne}{l_{1}}
\newcommand{\SpeakerK}[1][k]{S_{#1}}
\newcommand{\ListenerK}[1][k]{L_{#1}}

\newcommand{\nullmsg}{\mathbf{0}}

%=====================================================================
%============================ annotations ============================

\let\oldmarginpar\marginpar
\renewcommand{\marginpar}[1]{\oldmarginpar[\color{red}\raggedright\scriptsize #1]{\color{red}\raggedright\scriptsize #1}}

\newcommand{\textnote}[1]{{\color{red}#1}}

%=====================================================================
%============================== helpers ==============================

\newcommand{\porq}{p \vee q}
\newcommand{\pandq}{p \wedge q}

\newcommand{\disjlexicon}[2]{
  \left[
    \begin{array}[c]{l@{ \ \mapsto \ } l}
      \porq    & \set{#1} \\
      \pandq   & \set{#2} \\
      \nullmsg & \set{w_{1}, w_{2}, w_{3}} \\
    \end{array}
  \right]}

\newcommand{\listenerMatrix}[6]{
  \begin{array}[c]{l *{4}{r}}
    \toprule
    #1 & w_{1} & w_{2} & w_{3} & w_{4} \\
    \midrule
    p        & #2 \\
    q        & #3 \\              
    \pandq    & #4 \\
    \porq   & #5 \\
    \nullmsg & #6 \\
    \bottomrule
  \end{array}}

\newcommand{\speakerMatrix}[5]{
  \begin{array}[c]{r *{5}{r}}
    \toprule
    #1 & p & q & \pandq & \porq & \nullmsg \\
    \midrule
    w_{1} & #2 \\
    w_{2} & #3 \\ 
    w_{3} & #4 \\ 
    w_{3} & #5 \\ 
    \bottomrule
  \end{array}}

\newcommand{\ListenerKMatrix}[4]{
  \begin{array}[c]{l *{3}{r}}
  \toprule
    #1 & w_{1} & w_{2} & w_{3} \\
    \midrule
    \LexStar  & #2 \\
    \Lex_{1}  & #3 \\
    \Lex_{2}  & #4 \\
    \bottomrule
  \end{array}}

\newcommand{\SpeakerKMatrix}[4]{
  \begin{array}[c]{l *{3}{r}}
    \toprule
    \Lex_{#1} & \porq & \pandq & \nullmsg \\
    \midrule
    w_{1}  & #2 \\
    w_{2}  & #3 \\
    w_{3}  & #4 \\
    \bottomrule
  \end{array}}

\newcommand{\smallhurfordlex}[3]{
  \setlength{\arraycolsep}{1pt}
  \left[
    \begin{array}[c]{l@{ \ \mapsto \ }r@{, \ } l@{ \ \mapsto \ }r@{, \ } l@{ \ \mapsto \ }r}
      A & \set{#1} &
      B & \set{#2} &
      X & \set{#3}
    \end{array}
  \right]}


\renewcommand{\smallhurfordlex}[3]{
  \setlength{\arraycolsep}{1pt}
  \left[
    \begin{array}[c]{l@{:\, }r@{,\, } l@{:\, }r@{,\, } l@{:\, }r}
      A & \set{#1} &
      B & \set{#2} &
      X & \set{#3}
    \end{array}
  \right]}

\definecolor{highlightcolor}{HTML}{D95F02}
\definecolor{annotationcolor}{HTML}{777777} %1B9E77}
\definecolor{worldinfocolor}{HTML}{E7298A}
\definecolor{lexcolor}{HTML}{D95F02}
\definecolor{costcolor}{HTML}{A6761D}

\renewcommand{\highlight}[1]{{\color{highlightcolor}#1}}

\newcommand{\lisZeroDef}{\listenerZero(\state \given \msg, \Lex) \propto \frac{\Indicator(\state \in \Lex(\msg))}{|\Lex(\msg)|}\Prior(\state)}
\newcommand{\spkOneDef}{\speakerOne(\msg \given \state, \Lex) \propto \exp\left(\log\left(\alpha\,\listenerZero(\state \given \msg, \Lex) \right)- \gamma\,\Costs(\msg)\right)}
\newcommand{\lisOneDef}{\listenerOne(\state \given \msg, \Lex) \propto \speakerOne(\msg \given \state, \Lex)\Prior(\state)}
\newcommand{\LisOneDef}{%
  \setlength{\arraycolsep}{1pt}%
  %\begin{array}[t]{r c l}
    \ListenerK[1](\state, \Lex \given \msg) = {\color{worldinfocolor}\ListenerK(\state \given \msg, \Lex)} {\color{lexcolor}\ListenerK(\Lex \given \msg)}% \\
    %\ListenerK(\Lex \given \msg) &\propto&  \Prior(\Lex) \sum_{\state\in\States} \SpeakerK(\msg \given \state, \Lex)\Prior(\state)
  %\end{array}
  }
\newcommand{\SpkTwoDef}{\SpeakerK[2](\msg \given \state, \Lex) \propto 
  \exp\left(
    \log
    \left({\color{worldinfocolor}\alpha\,\ListenerK[k-1](\state \given \msg, \Lex)}\right)
    -
    {\color{lexcolor}\beta \log\left(\ListenerK[k-1](\Lex\given\msg)\right)}
    -
    {\color{costcolor}\gamma\,\Costs(\msg)}
  \right)}
 

%%%%%%%%%%%%%%%%%%%%%%%%%%%%%%%%%%%%%%%%%%%%%%%%%%%%%%%%%%%%%%%%%%%%%%
%%%%%%%%%%%%%%%%%%%%%%%%%%%%%%%%%%%%%%%%%%%%%%%%%%%%%%%%%%%%%%%%%%%%%%

\begin{document}
\begin{poster}{
    % Show grid to help with alignment
    grid=false,
    % Column spacing
    colspacing=0.7em,
    % Color style
    headerColorOne=cyan!20!white!90!black,
    borderColor=cyan!30!white!90!black,
    % Format of textbox
    textborder=faded,
    % Format of text header
    eyecatcher=false,
    headerborder=open,
    headershape=roundedright,
    headershade=plain,
    background=none,
    bgColorOne=cyan!10!white,
    headerheight=0.11\textheight
}
{} % Eye Catcher
{Negotiating lexical uncertainty and expertise with disjunction\vspace{0.25em}}
{Roger Levy and Christopher Potts}
{\includegraphics[height=0.10\textheight]{stanford}\hspace{-20pt}\includegraphics[height=0.10\textheight]{ucsd}}

%%%%%%%%%%%%%%%%%%%%%%%%%%%%%%%%%%%%%%%%%%%%%%%%%%%%%%%%%%%%%%%%%%%%%%
\headerbox{Communicating in language about language}{name=intro,column=0,row=0,span=2}{

  \begin{itemize}\setlength{\itemsep}{0pt}
  \item Languages are neither fixed across time nor identically
    reproduced in all speakers, but rather continually renegotiated
    during interactions.

  \item People accommodate to each other's usage patterns, form
    temporarily lexical pacts, and instruct each other about their
    linguistic views.

  \item Some of this communication in language about language is
    direct, as with explicit definitions, but much of it arrives via
    secondary pragmatic inferences.

  \item Disjunction supports what appear to be opposing inferences
    about language.

    \begin{itemize}\setlength{\itemsep}{0pt}
    \item \textbf{Hurfordian pressure:} \word{X or Y} conveys that \word{X} and
      \word{Y} are disjoint
    \item \textbf{Definitional inference:} \word{X or Y} conveys that
      \word{X} and \word{Y} are synonymous
    \end{itemize}
    
  \item Cross-linguistically robust, so we seek a single pragmatic
    model that can derive both of these meanings from the semantics of
    disjunction given different contextual assumptions.
  \end{itemize}
}

%%%%%%%%%%%%%%%%%%%%%%%%%%%%%%%%%%%%%%%%%%%%%%%%%%%%%%%%%%%%%%%%%%%%%%
\headerbox{Hurfordian perceptions and intentions}{name=hurford,column=0,row=2,span=2,below=intro}{

  \textbf{Generalization}: \word{X or Y} conveys that the speaker is
  using a lexicon where \word{X} and \word{Y} are disjoint, or 
  addresses a speaker concern that the listener is using such a 
  lexicon.

  \vspace{4pt}

  \begin{minipage}[c]{0.48\linewidth}
    \begin{examples}\setlength{\itemsep}{0pt}
    \item some of our \highlight{American or Californian} friends
    %\item Stop discrimination of an \highlight{applicant or person}
    %  due to their tattoos.
    %\item Promptly report any \highlight{accident or occurrence}.
    \item the \highlight{canoe or boat} will be held by the stream's
      current
    % \item As an \highlight{actor or performer}, you are always
    %   worried about what the next job's going to be.
    % \item After the loss of the \highlight{animal or pet}, \ldots    
    % \item the effect was \highlight{greater than, or not equal to,}
    %   the cause.
    \item In 1940, 37\% of us had gone to a \highlight{church or
        synagogue} in the last week.
    \end{examples}

    \begin{center}
      \begin{tabular}[c]{r l}
        \toprule
        \multicolumn{2}{c}{\textbf{Our corpus}} \\
        \midrule
        general or specific & 75 \\
        specific or general & 86 \\
        \bottomrule
      \end{tabular}
    \end{center}       
  \end{minipage}
\hfill 
\begin{minipage}[c]{0.48\linewidth}  
  \includegraphics[width=1\textwidth]{../fig/chemla.pdf}

  \word{X or Y} usage correlates with \word{X} implicating \word{not Y}
\end{minipage}
}

%%%%%%%%%%%%%%%%%%%%%%%%%%%%%%%%%%%%%%%%%%%%%%%%%%%%%%%%%%%%%%%%%%%%%%
\headerbox{Disjunctive definition and identification}{name=def,column=0,row=2,span=2,below=hurford}{
  \textbf{Generalization}: can arise when speaker and listener are 
  willing to coordinate on the speaker's lexicon

  \vspace{6pt}

  \begin{minipage}[c]{0.48\linewidth}
    \begin{examples}\setlength{\itemsep}{0pt}
    \item wine lover or \emph{oenophile}
    \item A Geological History of Manhattan or New York Island
    \item New Haven or ``the Elm City''   
    \item woodchuck or ``land beaver''
    \end{examples}    
  \end{minipage}
  \hfill 
  \begin{minipage}[c]{0.48\linewidth}
    \begin{itemize}\setlength{\itemsep}{0pt}
    \item Motivation: speaker is a known expert and `instructor'; listener is a known non-expert
    \item Motivation: speaker wishes to display expertise to another expert
    \item Motivation: speaker sees value in (temporarily or permanently) defining a term
    \end{itemize}    
  \end{minipage}
  
  \vspace{6pt}

  Attested in Chinese, German, Hebrew, Ilokano, Japanese, Russian, and
  Tagalog. Seems to survive even where the language has a dedicated
  definitional disjunction morpheme (e.g., Finnish, Italian).  

}

%%%%%%%%%%%%%%%%%%%%%%%%%%%%%%%%%%%%%%%%%%%%%%%%%%%%%%%%%%%%%%%%%%%%%%
\headerbox{Further information}{name=info,column=0,row=3,span=2,below=def}{  
  
  Paper, references, model code, corpus data: \url{http://github.com/cgpotts/pypragmods/}
}

%%%%%%%%%%%%%%%%%%%%%%%%%%%%%%%%%%%%%%%%%%%%%%%%%%%%%%%%%%%%%%%%%%%%%%
\headerbox{Modeling communication with anxious experts}{name=model,column=2,row=0,span=2}{

  \newcommand{\labelednode}[2]{\put(#1){\makebox(0,0)[l]{#2}}}
  \newcommand{\annotation}[2]{\labelednode{#1}{{\footnotesize #2}}}
  \newcommand{\picarrow}[3][1.8]{\put(#2){\vector(#3){#1}}}

  \setlength{\unitlength}{1cm}
  \begin{picture}(16,7.25)

    \labelednode{2,7.2}{$\ldots$}
    \picarrow{2, 7.1}{-2,-1}

    
    \annotation{4,6.45}{{\color{worldinfocolor}world information} $-$ {\color{lexcolor}lexical preferences} $-$ {\color{costcolor}costs}}

    \labelednode{0,5.95}{$\SpkTwoDef$}
    \picarrow{0.2, 5.75}{2,-1}
        
    \annotation{4,4.2}{{\color{worldinfocolor}world information} $*$ {\color{lexcolor}lexical discrimination}}

    \labelednode{2,4.7}{$\LisOneDef$}
    \picarrow[1.3]{2.2, 4.5}{0,-1}
   
    \labelednode{2,2.95}{$\lisOneDef$}
    \picarrow{2, 2.75}{-2,-1}
    
    \labelednode{0,1.7}{$\spkOneDef$}
    \picarrow{0.2, 1.5}{2,-1}
    
    \labelednode{2,0.25}{$\lisZeroDef$}

     {\color{annotationcolor}      
      \labelednode{12.25,4.7}{\parbox{3.6cm}{suffices for manner implicature and embedded scalar implicature}}
      \picarrow[1]{12.05, 4.7}{-1,0}
      \labelednode{12.25,2.95}{\parbox{3.6cm}{suffices for unembedded scalar implicature}}
      \picarrow[1]{12.05, 2.95}{-1,0}
    }
  \end{picture}
}

%%%%%%%%%%%%%%%%%%%%%%%%%%%%%%%%%%%%%%%%%%%%%%%%%%%%%%%%%%%%%%%%%%%%%%
\headerbox{Hurfordian contexts}{name=hurfordmodel,column=2,row=1,span=2,below=model}{  
  With high disjunction costs, exclusivization maximizes the justification for the long form.

  \vspace{4pt}

  
  \newcommand{\lisMat}[4]{
    {\tiny
    \setlength{\arraycolsep}{1pt}
    \begin{array}[c]{l *{3}{r}}
      \toprule
      #1 & w_{1} & w_{2} & w_{1}{\vee}w_{2} \\
      \midrule
      A & #2\\
      B & #3 \\
      C & #4 \\
      \bottomrule
    \end{array}}}

  \newcommand{\spkMat}[4]{
    {\tiny
    \setlength{\arraycolsep}{1pt}
    \begin{array}[c]{l *{3}{r}}
      \toprule
      #1 & A & B & C\\
      \midrule
      w_{1} & #2\\
      w_{2} & #3 \\
      w_{1}{\vee}w_{2} & #4 \\
      \bottomrule
    \end{array}}}


\setlength{\tabcolsep}{2pt}
\begin{tabular}[c]{cc}  
  $\begin{array}[c]{c@{ \ \leftarrow \ } c @{ \ \leftarrow \ } c}     
     \listenerZero  & \speakerOne & \listenerOne \\
    \lisMat{\LexStar}{0&0&0}{0&0&0}{0&0&0} & \spkMat{\LexStar}{0&0&0}{0&0&0}{0&0&0} & \lisMat{\LexStar}{0&0&0}{0&0&0}{0&0&0} \\[4ex]
    \lisMat{\Lex_{1}}{0&0&0}{0&0&0}{0&0&0} & \spkMat{\Lex_{1}}{0&0&0}{0&0&0}{0&0&0} & \lisMat{\Lex_{1}}{0&0&0}{0&0&0}{0&0&0} \\[4ex]
    \lisMat{\Lex_{2}}{0&0&0}{0&0&0}{0&0&0} & \spkMat{\Lex_{2}}{0&0&0}{0&0&0}{0&0&0} & \lisMat{\Lex_{2}}{0&0&0}{0&0&0}{0&0&0}   
   \end{array}$
  $\leftarrow\cdots$
 &
  \begin{minipage}[c]{0.48\textwidth}
    \setlength{\arraycolsep}{2pt}
    $\begin{array}[c]{l l r r r}
       \toprule
       & \ListenerK[3] \text{ hears } \word{A or X}       & w_{1} & w_{2} & w_{1}{\vee}w_{2} \\
       \midrule
       \LexStar & \smallhurfordlex{w_{1}}{w_{2}}{w_{1}, w_{2}} & 0 & 0 & 0.16 \\[1ex]
       \Lex_{1} & \smallhurfordlex{{\color{lexcolor}w_{1}}}{w_{2}}{{\color{lexcolor}w_{2}}} & 0 & 0 & \graycell{0.47} \\[1ex]
       \Lex_{2} & \smallhurfordlex{w_{1}}{w_{2}}{w_{1}} & 0 & 0 & 0.38 \\
       \bottomrule
     \end{array}$
     \phantom{a}~\hfill  
     $\alpha = 2$; 
     $\beta = 1$; 
     $\Costs(\word{or}) = 1$
   \end{minipage} \\[14ex]
  Lexicon-specific agents & Joint world--lexicon listener
  \end{tabular}
}

%%%%%%%%%%%%%%%%%%%%%%%%%%%%%%%%%%%%%%%%%%%%%%%%%%%%%%%%%%%%%%%%%%%%%%
\headerbox{Definitional contexts}{name=defmodel,column=2,row=2,span=1,below=hurfordmodel}{

  Require low disjunction costs and high $\beta$: the speaker is
  invested in communicating about the lexicon and can tolerate the
  cost of a disjunction that is synonymous with one of its disjuncts.

  \vspace{4pt}
    
  \setlength{\arraycolsep}{2pt}
   $\begin{array}[c]{l@{ }l r r r}
    \toprule
      &\ListenerK[3] \text{ hears } \word{A or X}  & w_{1} & w_{2} & w_{1}{\vee}w_{2} \\
    \midrule
    \LexStar & \smallhurfordlex{w_{1}}{w_{2}}{w_{1}, w_{2}} & 0 & 0 & 0 \\[1ex]
    \Lex_{2} & \smallhurfordlex{w_{1}}{w_{2}}{w_{2}}        & 0 & 0 & 0 \\[1ex]
    \Lex_{3} & \smallhurfordlex{{\color{lexcolor}w_{1}}}{w_{2}}{{\color{lexcolor}w_{1}}}  & \graycell{.88} & 0 & .12\\
    \bottomrule
  \end{array}$  
  \phantom{a}~\hfill
  $\alpha = 5$; 
  $\beta = 7$; 
  $\Costs(\word{or}) = 0.01$
}

%%%%%%%%%%%%%%%%%%%%%%%%%%%%%%%%%%%%%%%%%%%%%%%%%%%%%%%%%%%%%%%%%%%%%%
\headerbox{Characterization}{name=char,column=3,row=3,span=1,below=hurfordmodel}{  
  \includegraphics[width=\textwidth]{../fig/lex5-alpha-beta-gamma.png}

  Summarizes a search over many parameter settings using a large
  lexicon and large world space.

}



\end{poster}
\end{document}


