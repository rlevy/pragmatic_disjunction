\section{Introduction}\label{sec:intro}

Semantic acquisition is often thought of entirely in terms of children
learning to relate forms in their language to meanings (objects in the
world, social constructs and other abstract concepts;
\cite{Frank:Goodman:Tenenbaum:2009}). In practice, however, semantic
acquisition is a lifelong process involving a constant interplay
between the languages of individuals and the language of the speech
community \cite{Lewis75LL}, and it is one that is often mediated
entirely by language, with connections to the world entering only
indirectly. This possibility is suggested by distributional approaches
to meaning, which hold that meaning is often inferrable from
linguistic context alone \cite{Harris54,TurneyPantel10}.

One way to use language to teach people about language is with
explicitly definitional sentences like \word{`Groundhog' means the
  same thing as `woodchuck'} or \word{`Groundhog' and `woodchuck' are
  synonymous}. Speakers can also leverage the presuppositions of
specific constructions to convey information about word meanings; for
instance, \cite{Hearst92} observes that constructions like \word{X
  such as Y} convey that \word{X} is a kind of (hyponym of) \word{Y}
(see also \cite{SnowEtAl05}), and thus one can learn relational
information from this construction even where one cannot ground
\word{X} or \word{Y} in the world.

These phenomena are important for understanding the semantics of
natural language, but they are perhaps not surprising from a
communicative perspective, since they involve constructions that
unambiguously convey definitional information.  However, many signals
about the conventions of language are less direct, encoded instead via
side effects of the central message.  Our focus in this paper is on
disjunctive statements used to convey, parenthetically, information
about word meanings. In these \tech{definitional disjunctions}, the
speaker uses one disjunct to define the other, as in examples like
\word{wine lover or oenophile} and \word{oenophile or wine lover}.
Such examples are striking because disjunction prototypically involves
words with conceptually related but disjoint terms, whereas here it is
used for terms with identical meanings.  The contextual circumstances
that support such readings can be characterized in broad terms as
follows: it is mutual, public knowledge of the speaker and the hearer
that the speaker is an expert in the relevant domain and the listener
is not (or cannot be presumed to be), and the speaker has an interest
in conveying information about the language itself (say, for
pedagogical purposes, or to make later discourse easier to conduct).

The puzzle deepens when we see that definitional disjunction is not a
quirk of English that we could perhaps blame on a nonce ambiguity.  We
know it is attested in a wide range of typologically and
geographically diverse languages: Chinese, Finnish, German, Hebrew,
Ilokano, Japanese, Russian, and Tagolog.  This suggests that it would
be a mistake to treat definitional uses as an idiosyncratic lexical
ambiguity of disjunctive coordination. Rather, they must arise from
the core meaning of disjunction.  The present paper develops a
recursive Bayesian models of linguistic communication
\cite{Franke09DISS,Jaeger:2011,Frank:Goodman:2012} on which
definitional disjunctions arise naturally in contexts in which it is
mutual, public knowledge between speaker and listener that the speaker
is an authority on the relevant word meanings and the listener is not.
The model builds on insights from the lexical uncertainty model of
\cite{Bergen:Goodman:Levy:2012}, and is a direct extension of the
model of \cite{Smith:Goodman:Frank:2013} to account for the ways in
which speakers and listeners value conveying information about their
language as well as about the world. We find that pushing these models
to capture definitional disjunction leads to novel insights about the
nature of language use, pragmatic reasoning, and social cognition.

\mynote{In a ling paper, I'd have a section after this giving a lot of data on def.\ disj.}


%%% Local Variables: 
%%% mode: latex
%%% TeX-master: "definitional_disjunction"
%%% End: 

