\section{Introduction}\label{sec:intro}

Semantic acquisition is often thought of in terms of children learning
to relate forms in their language to meanings (objects in the world,
social constructs and other abstract concepts;
\citealt{Frank:Goodman:Tenenbaum:2009}). In practice, however, semantic
acquisition is a lifelong process involving a constant interplay
between the languages of individuals and the language of the speech
community \citep{Lewis75LL}, and it is one that is often mediated
entirely by language, with connections to the world entering only
indirectly. This possibility is suggested by distributional approaches
to meaning, which hold that meaning is often inferrable from
linguistic context alone \citep{TurneyPantel10}.

One way to use language to teach people about language is with
explicitly definitional sentences like \word{`Groundhog' means the
same thing as `woodchuck'} or \word{`Groundhog' and `woodchuck' are
synonymous}. Speakers can also leverage the presuppositions of
specific constructions to convey information about word meanings; for
instance, \citet{Hearst92} observes that constructions like \word{X
such as Y} convey that \word{X} is a kind of (hyponym of) \word{Y}
(see also \citealt{SnowEtAl05}), and thus one can learn relational
information from this construction even where one cannot ground
\word{X} or \word{Y} in the world.

These phenomena are important for understanding the semantics of
natural language, but they are perhaps not surprising from a
communicative perspective, since they involve constructions that
unambiguously convey definitional information.  Our focus in this
paper is on a construction is frequently used to convey definitional
information even though its core semantics seems to be at odds with
such uses.  In \tech{definitional disjunctions}, the speaker uses one
disjunct to define the other, as in examples like \word{wine lover or
oenophile} and \word{oenophile or wine lover}. Such examples are
striking because disjunction prototypically involves words with
conceptually related but disjoint terms, whereas here it is used for
terms with identical meanings.  

The puzzle deepens when we see that definitional disjunction is not a
quirk of English that we could perhaps blame on a nonce ambiguity.  We
know it is attested in a wide range of typologically and
geographically diverse languages: Chinese, Finnish, German, Hebrew,
Ilokano, Japanese, Russian, Tagolog.  This suggests that it would be a
mistake to treat definitional uses as an ambiguity. Rather, they must
arise from the core meaning of disjunction and general considerations
of language use, pragmatic reasoning, and social cognition.

The present paper develops a recursive Bayesian models of linguistic
communication \citep{Franke09DISS,Jaeger:2011,Frank:Goodman:2012} on
which definitional disjunctions arise naturally in contexts in which
it is mutual, public knowledge between speaker and listener that the
speaker is an authority on the relevant word meanings and the listener
is not.  The model builds on insights from the lexical uncertainty
model of \citet{Bergen:Goodman:Levy:2012}, and is a direct extension
of the model of \citet{Smith:Goodman:Frank:2013} to account for the
ways in which speakers and listeners value conveying information about
their language as well as about the world.

% Seems unweildly to try to do copular construcitions as well, but
% here's some prose to introduce them:
%
% In \tech{definitional copular constructions}, the speaker defines the
% subject in terms of the object, or vice versa, as in examples like
% \word{Wine lovers are oenophiles} and \word{Oenophiles are wine
% lovers}.  Such examples are striking because predictions of this sort
% are generally used to ascribe contingent properties to objects,
% whereas definitional uses convey purely language-internal information
% and in fact are presumed not to ascribe new properties.


















