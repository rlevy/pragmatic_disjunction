\documentclass[12pt]{article}

\usepackage{setspace}
\usepackage{times}
\usepackage[left={1in}, right={1in}, top={1in}, bottom={1in}, head={0in}, foot={0.25in}]{geometry}
\usepackage{natbib}
\usepackage{fancyhdr}
\usepackage[compact]{titlesec}
\usepackage{url}
\usepackage{paralist}

\usepackage{enumitem}
\setlist{topsep=2pt,itemsep=0pt,parsep=0pt}


\newcommand{\word}[1]{\emph{#1}}
\newcommand{\tech}[1]{\emph{#1}}
\newcommand{\set}[1]{\ensuremath{\left\{#1\right\}}}
\newcommand{\posscitet}[1]{\citeauthor{#1}'s~(\citeyear{#1})}

\pagestyle{fancy}
\fancyhead{}
\lhead{\textbf{Communicating in language, and about language, using disjunction}}
\rhead{LSA 2015}
\cfoot{}
\rfoot{\textbf{Word count: 493}}
\renewcommand{\headrulewidth}{0pt}
\renewcommand{\baselinestretch}{0.96}

\begin{document}

\paragraph{Pr{\'e}cis}
There is a well-known preference for disjunctions \word{X or Y} to be
construed so that X and Y are semantically disjoint, and violations
stimulate implicatures.  However, there are two types of frequent and
felicitous disjunctions that violate this preference preference:
disjunctions of terms in a one-way semantic inclusion relation such as
\word{boat or canoe}, and disjunctions of synonymous terms like
\word{wine lover or \textbf{oenophile}}, which are often felicitously
used to convey definitional information.  Here we show how both these
classes are naturally predicted by a novel recursive probabilistic
model of communication in which speakers and listeners simultaneously
exchange information about the world and about the language they are
using.

\paragraph{Disjointness implicatures}
\posscitet{Hurford:1974} generalization (HG) states the preference for
disjuncts to present mutually exclusive options. In actual usage, HG
is routinely violated (\citealt{Rohdenburg:1985}; for large corpora of
naturally occurring violations, see
\citealt{Potts13MICH,Chemla-HurfordCounts}).  We suggest that these
examples occur when the speaker includes the specific term to block
one of two types of implicatures that could arise from the general
term alone:

\begin{itemize}
\item \tech{I-implicature}: when the listener might 
  construe the general term as \emph{limited to} a salient or prototypical
  subkind. (Where \word{boat} tends to identify motorboats, the
  speaker uses \emph{boat or canoe} to ensure that
  canoes are included.)

\item \tech{Q-implicature}: where failure to explicitly mention a
  salient or prototypical subkind might trigger an ad-hoc scalar
  implicature of its \emph{exclusion} \citep{hirschberg:1985}. (The
  speaker uses \emph{swimwear or bikini} to ensure that bikinis are included.)
\end{itemize}

\paragraph{Definitional readings} Very different effects arise for
disjunctions of synonymous terms \citep{Horn89,Rohdenburg:1985}, as in
\word{wine lover or \textbf{oenophile}} and \word{\textbf{oenophile}
  or wine lover}. Such readings seem meta-linguistic, in that the
speaker seems to be teaching the listener about a word meaning. For
such readings to arise, it must be mutual, public knowledge that the
speaker is an expert in the relevant domain and the listener is not
(or cannot be presumed to be), and that the speaker has an interest in
conveying information about the language itself.

\paragraph{Model} The above pattern is attested in typologically and
geographically diverse languages, suggesting that it derives from the
basic semantics of disjunction. We capture it within a single
probabilistic model of pragmatic reasoning. The model's crucial
components are that it (i) allows a flexible relationship between
literal meanings and context-specific ``refined'' meanings, (ii)
preserves refined meanings through semantic composition, and (iii)
lets speakers communicate, not just about the world, but also about
the language \citep{Smith:Goodman:Frank:2013}.
\citet{bergen-levy-goodman:2014} show how (i)--(ii) suffice to account
for the Q- cases of disjointness described above; we show how they
also suffice to account for the I- cases.  We show that adding (iii)
preserves these results and permits definitional disjunction to arise
under precisely the circumstances described above. On this analysis,
the known disjunct serves as a \tech{focal point} that the speaker and
listener can coordinate on for the meaning of the unknown disjunct.

%=====================================================================
% Efficient inline bibliography style from 
% http://user.astro.columbia.edu/~williams/mn2ebst/

\let\olditem\item
\renewenvironment{thebibliography}[1]{
\footnotesize
  \paragraph{References}
  \let\par\relax\let\newblock\relax
  \renewcommand{\item}[1][]{\olditem[\textbullet]}%
  \inparaenum
}{\endinparaenum}

\bibliographystyle{apalike}
\begin{spacing}{0.85}
\bibliography{levy-potts-lsa2015-bib}
\end{spacing}

\end{document}

