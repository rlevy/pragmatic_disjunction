\documentclass[12pt]{article}

\usepackage{times}
\usepackage[left={1in}, right={1in}, top={1in}, bottom={1in},head={0in}]{geometry}
\usepackage{natbib}
\usepackage{fancyhdr}
\usepackage[compact]{titlesec}
\usepackage{url}
\usepackage{paralist}

\usepackage{enumitem}
\setlist{topsep=2pt,itemsep=0pt,parsep=0pt}


\newcommand{\word}[1]{\emph{#1}}
\newcommand{\tech}[1]{\emph{#1}}
\newcommand{\set}[1]{\ensuremath{\left\{#1\right\}}}
\newcommand{\posscitet}[1]{\citeauthor{#1}'s~(\citeyear{#1})}

\pagestyle{fancy}
\fancyhead{}
\lhead{\textbf{Communicating in language, and about language, using disjunction}}
\rhead{LSA 2015}
\cfoot{}
\renewcommand{\headrulewidth}{0pt}


\begin{document}

\paragraph{Pr{\'e}cis}
There is a well-known preference for disjunctions \word{X or Y} to be
construed so that \word{X} and \word{Y} are semantically disjoint, and
violations stimulate implicatures.  However, disjunctions of
synonymous terms like \word{wine lover or \textbf{oenophile}}, which
maximally violate the disjointness preference, are often felicitously
used to convey definitional information. We develop a recursive
probabilistic model of communication in which speakers and listeners
simultaneously exchange information about the world and about the
language they are using, and we show that this model predicts both the
disjointness implicatures and the possibility of definitional
disjunctions.

\paragraph{Disjointness implicatures}
\posscitet{Hurford:1974} generalization (HG) states
the preference for disjuncts to present mutually exclusive options. In
actual usage, HG is routinely violated (\citealt{Rohdenburg:1985}; for
large corpora of naturally occurring violations, see
\citealt{Potts13MICH,Chemla-HurfordCounts}).  However, the truth
behind HG is that such examples generate implicatures
that can be understood without recourse to HG:

\begin{itemize}
\item \tech{I-implicature}: the speaker worries that the listener will
  construe a general term as excluding a subkind due to salience or
  prototypicality. (Where \word{boat} tends to identify motorboats,
  the speaker might disjoin \set{\word{canoe}, \word{boat}} to ensure
  that canoes are included.)

\item \tech{Q-implicature}: the speaker worries that a specific term,
  which needs to be mentioned for some reason, will be construed
  exhaustively. (E.g., the speaker disjoints \set{\word{boss},
    \word{supervisor}} to highlight bosses while still giving the
  statement the needed generality.)
\end{itemize}

\paragraph{Definitional readings} Very different effects arise for
disjunctions of synonymous terms \citep{Horn89,Rohdenburg:1985}, as in
\word{wine lover or \textbf{oenophile}} and \word{\textbf{oenophile}
  or wine lover}. Such readings seem meta-linguistic, in that the
speaker seems to be teaching the listener about a word meaning. For
such readings to arise, it must be mutual, public knowledge that the
speaker is an expert in the relevant domain and the listener is not
(or cannot be presumed to be), and that the speaker has an interest in
conveying information about the language itself.

\paragraph{Model} The above pattern is attested in typologically and
geographically diverse languages, suggesting that it derives from the
basic semantics of disjunction. We capture it within a single
probabilistic model of pragmatic reasoning. The models' crucial
components are that it (i) allows a flexible relationship between
literal meanings and context-specific ``refined'' meanings, (ii)
preserves refined meanings through semantic composition, and (iii)
lets speakers communicate, not just about the world, but also about
the language
\citep{Smith:Goodman:Frank:2013}. \citet{Bergen:Goodman:Levy:2012}
show that (i)--(ii) suffice for deriving the Q- and I- implicatures
discussed above. We show that adding (iii) preserves these results and
permits definitional disjunction to arise under precisely the
circumstances described above. On this analysis, the known disjunct
serves as a \tech{focal point} that the speaker and listener can
coordinate on for the meaning of the unknown disjunct.

%=====================================================================
% Efficient inline bibliography style from 
% http://user.astro.columbia.edu/~williams/mn2ebst/

\let\olditem\item
\renewenvironment{thebibliography}[1]{\footnotesize
  \paragraph{References}
  \let\par\relax\let\newblock\relax
  \renewcommand{\item}[1][]{\olditem[\textbullet]}%
  \inparaenum}{\endinparaenum}

\bibliographystyle{apalike}
\bibliography{levy-potts-lsa2015-bib}

\end{document}

