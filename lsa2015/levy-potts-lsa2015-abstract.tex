\documentclass[12pt]{article}

\usepackage{times}
\usepackage{geometry}
\usepackage{natbib}
\usepackage{fancyhdr}

\newcommand{\word}[1]{\emph{#1}}
\newcommand{\set}[1]{\ensuremath{\left\{#1\right\}}}
\newcommand{\posscitet}[1]{\citeauthor{#1}'s~(\citeyear{#1})}

\pagestyle{fancy}
\fancyhead{}
\lhead{TITLE}
\rhead{}
\cfoot{}
\renewcommand{\headrulewidth}{0pt}


\begin{document}

\begin{itemize}

\item
There are well-known communicative pressures on disjunctive phrases of
the form \word{X or Y} to be construed so that the meanings of
\word{X} and \word{Y} are disjoint. 

\item
The most widely studied instance of this is that \word{X or Y} is
generally taken to conversationally implicate that its conjunctive
counterpart \word{X and Y} is pragmatically inaccessible (false,
inappropriate, irrelevant, unknown to the speaker, etc.). The two
disjuncts are thus presented as non-overlapping options.

\item
\posscitet{Hurford:1974} generalization (HG) is a direct statement of the
overall communicative preassure. It says that, subject to certain
exceptions, \word{X or Y} is felicitious only if the meanings of
\word{X} and \word{Y} are disjoint. 

\item
In actual usage, HG is routinely violated, suggesting that it is not a
separate constraint on forms. We have collected a curated corpus of
161 violations involving phrases like \word{canoe or boat},
\word{Paris or France}, \word{China or Bejing}, \word{writer or
  playwright} and \word{tired or exhausted}. See also
\citet{Chemla-HurfordCounts} for quantative study of similar patterns.

\item
However, the truth behind HG is that these examples reliably generate
conversational implicatures of their own, suggesting that there is 
something to be explained. We divide these implicatures into two 
classes:

\begin{itemize}

\item
I-implicature: The speaker is worried that a general term will be
construed as excluding one of the subkinds due to salience or
prototypicality. For example, at a marina where motorboats are
prevalent, so that \word{boat} often picks out just motorboats, a
speaker might use a disjunction of \set{\word{boat}, \word{boat}} in
order to ensure that the listener knows that the speaker intends to
include canoes.

\item
Q-implicature: The speaker is worried that a specific term, which
needs to be mentioned for salience of some other reason, will be
construed exhaustively. For example, \word{19\% of employees dated a
  boss or supervisor}, where mentioning bosses helps the listener
to recognize the speaker's intententions, even though the statement
need sot cover a broader group.

\end{itemize}

\item 
The pressure for \word{X or Y} to involve semantically disjoint
\word{X} and \word{Y} is maximally violated in examples that we call
\word{definitional disjunction} \citep{Horn89,Rohdenburg:1985}, as in
examples like \word{wine lover or oenophile} and \word{oenophile or
  wine lover}. Such readings have been described as meta-linguistic,
in that the speaker seems to be trying to teach the listener something
about the conventions of the language. The contextual circumstances
that support such readings can be characterized in broad terms as
follows: it is mutual, public knowledge of the speaker and the hearer
that the speaker is an expert in the relevant domain and the listener
is not (or cannot be presumed to be), and the speaker has an interest
in conveying information about the language itself (say, for
pedagogical purposes, or to make later discourse easier to conduct).

\item
Summarizing the empirical picture, we find a tension. On the one hand,
\word{X or Y} is typically construed (and intended) as involving
semantically disjoint \word{X} and \word{Y}, and that violations of
this at the level of literal content give rise to rich conversational
implicatures. On the other hand, disjunction can be used to signal
that the speaker regards the two disjuncts as synonymous, a direct
countervention of the broader norms.

\item
The puzzle deepens when we see that the empirical picture, including
the definitional disjunction readings, is not a quirk of English that
we could perhaps blame on a nonce ambiguity. We know it is attested in
a wide range of typologically and geographically diverse languages:
Chinese, Finnish, German, Hebrew, Ilokano, Japanese, Russian, and
Tagolog. This suggests that the full range of readings derives from
the literal semantics of disjunction and systematic pragmatic
pressures.

\item
We seek to capture the full range of behavior within a single
recursive Bayesian model of pragmatic reasoning. These models find
their conceptual origins \posscitet{Lewis69} work on signaling
systems, and their technical details build on ideas the iterated best
response models of \citet{Jaeger:2007} and \citet{Franke09DISS}. They
have been shown to achieve tight correlations with experimental data
(e.g., \citealt{Frank:Goodman:2012}) and to contribute to artificial
agents that communicate effectively with each other to solve a
collaborative task \citep{Vogel-etal:2013}.

\item
We extend this basic model in two ways, building on insights of
\citet{Bergen:Goodman:Levy:2012} and \citet{Smith:Goodman:Frank:2013}:
(i) we directly model the uncertainty that speakers have about the
evolving conventions of their language, and (ii) we allow that
speakers might desire to communicate, not just information about the
world, but also simultaneously information about their conception of
the language. 

\item
Within this model, the different readings naturally emerge depending
on specific contextual parameters relating to speaker expertise,
listener malleability, and information contained in the common ground.
The basic converational implicature (exluding \word{and}) was already
derived by \citealt{Bergen:Goodman:Levy:2012}; our contribution is to
extend this to the effects of HG ``violations'' and to characterizing
where and how definitional disjunction arises.

\end{itemize}

\renewcommand{\bibsection}{\paragraph{References}}
\bibliographystyle{apalike}
\bibliography{levy-potts-lsa2015-bib}

\end{document}


