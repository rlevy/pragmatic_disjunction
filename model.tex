\section{Model}\label{sec:model}

Communication games model production and interpretation.  

\begin{definition}[Communication games]\label{def:struc} 
  A communication game is a structure $\Cgame$:
  \begin{enumerate}\setlength{\itemsep}{0pt}
  \item $\States$ is a set of states (worlds, referents, propositions, etc.).
  \item $\Messages$ is a set of messages.
  \item $\Lex: \Messages \mapsto \wp(\States)$ is a lexicon.
  \item $\Prior : \States \mapsto [0,1]$ is a prior probability
    distribution over states.   
  \item $\Costs : \Messages \mapsto \Reals$ is a cost function on messages.
  \end{enumerate}
\end{definition}

The speaker observes a state $\state$ and chooses a message $\msg$
based on $\state$ and the message costs $\Costs$. The listener chooses
a state $\state'$ based on $\msg$ and her prior expectations $\Prior$
about the state.  We assume throughout that the speaker will choose a
message that has maximum probability given the state, and that the
hearer will choose a state that has maximum probability given the
speaker's message.  This amounts to the assumption that the speaker
and hearer would like to communicate, which might or might not be tied
up with their real-world goals
\cite{Franke-etal:2009,Asher:Lascarides:2013}

Production and interpretation can be modeled as a recursive process in
which the speaker and listener reason about each other reasoning about
each other.  Suppose we begin with a truth-conditional listener
$\GenericListener$: given message $\msg$, $\GenericListener$ guesses a
state $\state \in \Lex(\msg)$ based only on the prior.  The speaker
$\GenericSpeaker$ can model this truth-conditional listener in
production, anticipating her inferences and trying to respond in a way
that maximizes the chances of successful communication.  What's more,
$\GenericListener$ can plan for this kind of sophisticated speaker and
make her inferences accordingly.  And so forth. The process can
proceed until the system stabilizes or until the participants reach
the limits of their rationality, mental energy, or commitment to the
cause
\cite{CamererHo:2004,Franke:2008,Franke09DISS,Jaeger:2007,Jaeger:2011}.

The most basic listener $\listenerZero$ simply uses the
truth-conditions of the message $\msg$ and the prior over states to
estimate the probability of each state:

\begin{definition}[$\listenerZero$]\label{def:l0}
  \[
  \listenerZero(\state \given \msg, \Lex) 
  \propto
  \frac{\mathbb{I}(\state \in \Lex(\msg)}{|\Lex(\msg)|}
  \Prior(\state)
  \]
\end{definition}

The most basic speaker $\speakerOne$ reasons about $\listenerZero$,
taking message costs into account:

\begin{definition}[$\speakerOne$]\label{def:s1}  
  \[
  \speakerOne(\msg \given \state, \Lex) 
  \propto
  \exp
  \left(
    \log
    \left(\listenerZero(\state \given \msg, \Lex)\right) 
    - \Costs(\msg)
  \right)
  \]
\end{definition}

The $\listenerOne$ listener responds to $\speakerOne$ in the sense
that it reasons about that agent and the priors; the definition is
parallel to the one for $\listenerZero$ except that the starting point
is the pragmatic distribution $\speakerOne$ rather than the truth
conditions given directly by $\Lex$:

\begin{definition}[$\listenerOne$]\label{def:l1}
  \[
  \listenerOne(\state \given \msg, \Lex) 
  \propto 
  \speakerOne(\msg \given \state, \Lex)\Prior(\state)
  \]
\end{definition}

The above are all lexicon-specific in the sens that the basic semantic
interpretation function $\Lex$ remains fixed throughout all the
calculations.  The agent $\ListenerOne$ is the first in the hierarchy
to reason in terms of $\Lex$; this listener is identical to the
``social anxiety'' listener of \cite{Smith:Goodman:Frank:2013}. It
forms a posterior over lexica, and marginalizes over this posterior to
form beliefs about communicated meaning:

\begin{definition}[$\ListenerOne$]\label{def:l1}
  \[
  \ListenerOne(\state, \Lex \given \msg) 
  = 
  \sum_{\Lex} \listenerOne(\state \given \msg, \Lex) \ListenerOne(\Lex \given \msg) 
  \]
  where
  $\ListenerOne(\Lex \given \msg) \propto \Prior(\Lex) \sum_{\state\in\States} \speakerOne(\msg \given \state)\Prior(\state)$
\end{definition}

The novelty in our approach comes from treating the expert speaker
differently. This speaker does not have social anxiety, but rather
knows the lexicon $\LexStar$ that they intend to use. Furthermore,
this speaker places value on the listener $\ListenerOne$'s correct
apprehension of $\LexStar$:

\begin{definition}[$\SpeakerK$]\label{def:s1}  
 DEFINE
\end{definition}

\begin{definition}[$\ListenerOne$]\label{def:l1}
  DEFINE
\end{definition}

\begin{figure}[htp]
  \centering
  \includegraphics[scale=1]{images/model}
  \caption{Pictoral representation of the recursive reasoning process.}
  \label{fig:model}
\end{figure}

%%% Local Variables: 
%%% mode: latex
%%% TeX-master: "definitional_disjunction"
%%% End: 


