%=====================================================================
%============================= packages ==============================

\usepackage[margin=1.0in]{geometry}
\usepackage{graphicx}
\usepackage{afterpage}
\usepackage{amssymb}
\usepackage{amsfonts}
\usepackage{amsmath}
\usepackage{amsthm}
\usepackage{natbib}
\renewcommand{\bibsection}{}
\bibpunct[:]{(}{)}{;}{a}{}{,}
\usepackage{xcolor}
\usepackage{colortbl}	
\usepackage{color}
\usepackage{array}
\usepackage[hang,flushmargin]{footmisc} 
\usepackage{footnote}
\usepackage[normalem]{ulem}
\usepackage{stmaryrd}
\usepackage{caption}
\usepackage{subcaption}
\usepackage{booktabs}
\definecolor{black}{rgb}{0,0,0}
\usepackage[colorlinks, linkcolor=black, urlcolor=black, citecolor=black]{hyperref}
\usepackage{gb4e}
\usepackage{tipa}
\noautomath

%=====================================================================
%========================= cross-references ==========================

% Flexible sec/fig/tbl/def cross-refs.
\newcommand{\Secref}[1]{Section~\ref{#1}}
\newcommand{\secref}[1]{Section~\ref{#1}}
\newcommand{\dashsecref}[2]{Sections~\ref{#1}--\ref{#2}}

\newcommand{\Figref}[1]{Figure~\ref{#1}}
\newcommand{\figref}[1]{Figure~\ref{#1}}
\newcommand{\dashfigref}[2]{Figures~\ref{#1}--\ref{#2}}
\newcommand{\Tabref}[1]{Table~\ref{#1}}
\newcommand{\tabref}[1]{Table~\ref{#1}}

% Examples:
\newcommand{\eg}[1]{(\ref{#1})}
\newcommand{\subeg}[2]{(\ref{#2})}
\newcommand{\dblsubeg}[3]{(\ref{#1}\ref{#2},~\ref{#3})}
\newcommand{\dashsubeg}[3]{(\ref{#1}\ref{#2}--\ref{#3})}

% In-text citations
\newcommand{\posscitet}[1]{\citeauthor{#1}'s (\citeyear{#1})}
\newcommand{\sposscitet}[1]{\citeauthor{#1}' (\citeyear{#1})}
\newcommand{\possciteauthor}[1]{\citeauthor{#1}'s}
\newcommand{\spossciteauthor}[1]{\citeauthor{#1}'}
\newcommand{\pgposscitet}[2]{\citeauthor{#1}'s (\citeyear{#1}:~#2)}
\newcommand{\secposscitet}[2]{\citeauthor{#1}'s (\citeyear{#1}:~$\S$#2)}
\newcommand{\pgcitealt}[2]{\citealt{#1}:~#2}
\newcommand{\seccitealt}[2]{\citealt{#1}:~$\S$#2}
\newcommand{\pgcitep}[2]{(\citealt{#1}:~#2)}
\newcommand{\seccitep}[2]{(\citealt{#1}:~$\S$#2)}
\newcommand{\pgcitet}[2]{\citeauthor{#1}~(\citeyear{#1}:~#2)}
\newcommand{\seccitet}[2]{\citeauthor{#1}~(\citeyear{#1}:~$\S$#2)}

%=====================================================================
%============================ text styles ============================

\newcommand{\word}[1]{\emph{#1}}
\newcommand{\tech}[1]{\textbf{#1}}
\definecolor{maroon}{HTML}{990000}
\newcommand{\highlight}[1]{{\color{maroon}#1}}
\newcommand{\ipa}[1]{\begin{tipa}#1\end{tipa}}

%=====================================================================
%============================== judgments ============================

\newcommand{\bad}{\sqz{${}^\ast$}}
\newcommand{\freebad}{${}^\ast$}
\newcommand{\marked}{\sqz{${}^\#$}}
\newcommand{\freemarked}{${}^\#$}

%=====================================================================
%=============================== model ===============================

\DeclareMathOperator{\exponential}{exp}

\newcommand{\tuple}[1]{\ensuremath{\left< #1 \right>}}
\newcommand{\set}[1]{\ensuremath{\left\{ #1 \right\}}}
\newcommand{\True}{\texttt{T}}
\newcommand{\False}{\texttt{F}}
\newcommand{\Reals}{\mathbb{R}}
\newcommand{\given}{\mid}
\newcommand{\Indicator}{\mathbb{I}}

\newcommand{\sem}[1]{\ensuremath{\llbracket#1\rrbracket}}
\newcommand{\States}{W}
\newcommand{\state}{w}
\newcommand{\Lex}{\mathcal{L}}
\newcommand{\LexStar}{\Lex^{\ast}}
\newcommand{\LexSet}{\mathbf{L}}
\newcommand{\Messages}{M}
\newcommand{\msg}{m}
\newcommand{\Costs}{C}
\newcommand{\Prior}{P}
\newcommand{\LexPrior}{P_{\LexSet}}

\newcommand{\listenerZero}{l_{0}}
\newcommand{\speakerOne}{s_{1}}
\newcommand{\listenerOne}{l_{1}}
\newcommand{\SpeakerK}[1][k]{S_{#1}}
\newcommand{\ListenerK}[1][k]{L_{#1}}

\newcommand{\nullmsg}{\mathbf{0}}

%=====================================================================
%============================ annotations ============================

\let\oldmarginpar\marginpar
\renewcommand{\marginpar}[1]{\oldmarginpar[\color{red}\raggedright\scriptsize #1]{\color{red}\raggedright\scriptsize #1}}

\newcommand{\textnote}[1]{{\color{red}#1}}

%=====================================================================
%============================== colors ===============================

\definecolor{lightgray}{HTML}{CCCCCC} 

\definecolor{highlightcolor}{HTML}{D95F02}
\definecolor{annotationcolor}{HTML}{777777} 
\definecolor{worldinfocolor}{HTML}{E7298A}
\definecolor{lexcolor}{HTML}{D95F02}
\definecolor{costcolor}{HTML}{A6761D}
\definecolor{defcolor}{HTML}{D95F02}
%\definecolor{hurfordcolor}{HTML}{00CC33}
\definecolor{hurfordcolor}{HTML}{1B9E77}
%\newcommand{\hurford}[1]{{\relax\color{hurfordcolor}#1}}
%\newcommand{\definitional}[1]{\relax{\color{defcolor}#1}}
\newcommand{\hurford}[1]{\mathbf{#1}}
\newcommand{\definitional}[1]{\mathbf{#1}}

\newcommand{\graycell}[1]{{\cellcolor[gray]{.8}#1}}

%=====================================================================
%============================== helpers ==============================

\newcommand{\porq}{p\,\word{or}\,q}
\newcommand{\pandq}{p\,\&\,q}

\newcommand{\disjlexicon}[2]{
  \left[
    \begin{array}[c]{l@{ \ \mapsto \ } l}
      \porq    & \set{#1} \\
      \pandq   & \set{#2} \\
      \nullmsg & \set{w_{1}, w_{2}, w_{3}} \\
    \end{array}
  \right]}

\newcommand{\listenerMatrix}[6]{
  \begin{array}[c]{l *{4}{r}}
    \toprule
    #1 & w_{1} & w_{2} & w_{3} \\
    \midrule
    p        & #2 \\
    q        & #3 \\              
    \pandq   & #4 \\
    \porq    & #5 \\
    \nullmsg & #6 \\
    \bottomrule
  \end{array}}

\newcommand{\speakerMatrix}[4]{
  \begin{array}[c]{r *{5}{r}}
    \toprule
    #1 & p & q & \pandq & \porq & \nullmsg \\
    \midrule
    w_{1} & #2 \\
    w_{2} & #3 \\ 
    w_{3} & #4 \\ 
    \bottomrule
  \end{array}}

\newcommand{\ListenerKMatrix}[4]{
  \begin{array}[c]{l *{3}{r}}
  \toprule
    #1 & w_{1} & w_{2} & w_{3} \\
    \midrule
    \LexStar  & #2 \\
    \Lex_{1}  & #3 \\
    \Lex_{2}  & #4 \\
    \bottomrule
  \end{array}}

\newcommand{\SpeakerKMatrix}[4]{
  \begin{array}[c]{l *{3}{r}}
    \toprule
    \Lex_{#1} & \porq & \pandq & \nullmsg \\
    \midrule
    w_{1}  & #2 \\
    w_{2}  & #3 \\
    w_{3}  & #4 \\
    \bottomrule
  \end{array}}

\newcommand{\smalldisjlex}[3]{
  \setlength{\arraycolsep}{1pt}
  \left[
    \begin{array}[c]{l@{ \ \mapsto \ }r@{, \ } l@{ \ \mapsto \ }r@{, \ } l@{ \ \mapsto \ }r}
      A & \set{#1} &
      B & \set{#2} &
      X & \set{#3}
    \end{array}
  \right]}

\newcommand{\smalldisjlexTargetDef}{\smalldisjlex{\definitional{w_{1}}}{w_{2}}{\definitional{w_{1}}}}

\newcommand{\smalldisjlexTargetHuford}{\smalldisjlex{\hurford{w_{1}}}{w_{2}}{\hurford{w_{2}}}}

%%%%%%%%%%%%%%%%%%%%%%%%%%%%%%%%%%%%%%%%%%%%%%%%%%%%%%%%%%%%%%%%%%%%%%
% Macros for the simple illustration of the model:

\newcommand{\lextoprule}{\specialrule{\heavyrulewidth}{1.5pt}{1.5pt}}
\newcommand{\lexmidrule}{\specialrule{\lightrulewidth}{1.5pt}{1.5pt}}
\newcommand{\lexbottomrule}{\specialrule{\heavyrulewidth}{1.5pt}{1.5pt}}

\newcommand{\examplelistener}[4][]{
  \setlength{\arraycolsep}{4pt}
  \begin{array}[c]{*{3}{r}}
    \lextoprule
    #1           & w_{1} & w_{2} \\
    \lexmidrule
    \word{cheap} & #2 \\
    \word{free}  & #3 \\
    \nullmsg     & #4 \\
    \lexbottomrule
  \end{array}}

\newcommand{\examplejoint}[4]{
  \setlength{\arraycolsep}{10pt}
  \begin{array}[c]{*{3}{r}}
    \lextoprule
    #1 & w_{1} & w_{2} \\
    \lexmidrule
    \LexStar & #2 \\
    \Lex_{1} & #3 \\
    \Lex_{2} & #4 \\
    \lexbottomrule
  \end{array}}

\newcommand{\examplelex}[2][]{\examplelistener[#1]{#2}{\True & \False}{\True & \True}}

\newcommand{\examplespeaker}[3][]{
  \setlength{\arraycolsep}{4pt}
  \begin{array}[c]{*{4}{r}}
    \lextoprule
    #1           & \word{cheap} & \word{free} & \nullmsg \\
    \lexmidrule
    w_{1} & #2 \\
    w_{2} & #3 \\
    \lexbottomrule
  \end{array}}

\newcommand{\exampleSpeaker}[3]{
  \setlength{\arraycolsep}{4pt}
  \begin{array}[c]{*{4}{r}}
    \lextoprule
    #1 & \word{cheap} & \word{free} & \nullmsg \\
    \lexmidrule
    w_{1} & #2 \\
    w_{2} & #3 \\
    \lexbottomrule
  \end{array}}

%%%%%%%%%%%%%%%%%%%%%%%%%%%%%%%%%%%%%%%%%%%%%%%%%%%%%%%%%%%%%%%%%%%%%%
% Macros for the closure example:

\newcommand{\closurelex}[6][1]{
  \renewcommand{\arraystretch}{1}
  \begin{array}[c]{*{8}{r}}
    \toprule
    &w_{1}&w_{2}&w_{3}&w_{1}{\vee}w_{2}&w_{1}{\vee}w_{3}&w_{2}{\vee}w_{3}&w_{1}{\vee}w_{2}{\vee}w_{3}\\
    \midrule
    p      & #2 \\
    q      & #4 \\
    \pandq & #3 \\
    \porq  & #5 \\
    \nullmsg & #6 \\
    \bottomrule
  \end{array}}

%%%%%%%%%%%%%%%%%%%%%%%%%%%%%%%%%%%%%%%%%%%%%%%%%%%%%%%%%%%%%%%%%%%%%%
% Macros for the Hurfordian and definitional examples

\newcommand{\lismat}[4]{
  \setlength{\arraycolsep}{1pt}
  \begin{array}[c]{l *{3}{r}}
    \toprule
    #1 & w_{1} & w_{2} & w_{1}{\vee}w_{2} \\
    \midrule
    A & #2\\
    X & #3 \\
    A\,\word{or}\,X & #4 \\
    \bottomrule
  \end{array}}

\newcommand{\spkmat}[4]{
  \setlength{\arraycolsep}{1pt}
  \begin{array}[c]{l *{3}{r}}
    \toprule
    #1 & A & X & A\,\word{or}\,X \\
    \midrule
    w_{1} & #2\\
    w_{2} & #3 \\
    w_{1}{\vee}w_{2} & #4 \\
    \bottomrule      
  \end{array}}                   

\renewcommand{\disjlexicon}[2]{
  \renewcommand{\arraystretch}{1}
  \left[   
    \begin{array}[c]{l@{ \ \mapsto \ } l}
      p   & \set{#1} \\
      q  & \set{#2} \\
    \end{array}
  \right]}


